% Options for packages loaded elsewhere
\PassOptionsToPackage{unicode}{hyperref}
\PassOptionsToPackage{hyphens}{url}
%
\documentclass[
]{article}
\usepackage{amsmath,amssymb}
\usepackage{iftex}
\ifPDFTeX
  \usepackage[T1]{fontenc}
  \usepackage[utf8]{inputenc}
  \usepackage{textcomp} % provide euro and other symbols
\else % if luatex or xetex
  \usepackage{unicode-math} % this also loads fontspec
  \defaultfontfeatures{Scale=MatchLowercase}
  \defaultfontfeatures[\rmfamily]{Ligatures=TeX,Scale=1}
\fi
\usepackage{lmodern}
\ifPDFTeX\else
  % xetex/luatex font selection
\fi
% Use upquote if available, for straight quotes in verbatim environments
\IfFileExists{upquote.sty}{\usepackage{upquote}}{}
\IfFileExists{microtype.sty}{% use microtype if available
  \usepackage[]{microtype}
  \UseMicrotypeSet[protrusion]{basicmath} % disable protrusion for tt fonts
}{}
\makeatletter
\@ifundefined{KOMAClassName}{% if non-KOMA class
  \IfFileExists{parskip.sty}{%
    \usepackage{parskip}
  }{% else
    \setlength{\parindent}{0pt}
    \setlength{\parskip}{6pt plus 2pt minus 1pt}}
}{% if KOMA class
  \KOMAoptions{parskip=half}}
\makeatother
\usepackage{xcolor}
\usepackage{color}
\usepackage{fancyvrb}
\newcommand{\VerbBar}{|}
\newcommand{\VERB}{\Verb[commandchars=\\\{\}]}
\DefineVerbatimEnvironment{Highlighting}{Verbatim}{commandchars=\\\{\}}
% Add ',fontsize=\small' for more characters per line
\newenvironment{Shaded}{}{}
\newcommand{\AlertTok}[1]{\textcolor[rgb]{1.00,0.00,0.00}{\textbf{#1}}}
\newcommand{\AnnotationTok}[1]{\textcolor[rgb]{0.38,0.63,0.69}{\textbf{\textit{#1}}}}
\newcommand{\AttributeTok}[1]{\textcolor[rgb]{0.49,0.56,0.16}{#1}}
\newcommand{\BaseNTok}[1]{\textcolor[rgb]{0.25,0.63,0.44}{#1}}
\newcommand{\BuiltInTok}[1]{\textcolor[rgb]{0.00,0.50,0.00}{#1}}
\newcommand{\CharTok}[1]{\textcolor[rgb]{0.25,0.44,0.63}{#1}}
\newcommand{\CommentTok}[1]{\textcolor[rgb]{0.38,0.63,0.69}{\textit{#1}}}
\newcommand{\CommentVarTok}[1]{\textcolor[rgb]{0.38,0.63,0.69}{\textbf{\textit{#1}}}}
\newcommand{\ConstantTok}[1]{\textcolor[rgb]{0.53,0.00,0.00}{#1}}
\newcommand{\ControlFlowTok}[1]{\textcolor[rgb]{0.00,0.44,0.13}{\textbf{#1}}}
\newcommand{\DataTypeTok}[1]{\textcolor[rgb]{0.56,0.13,0.00}{#1}}
\newcommand{\DecValTok}[1]{\textcolor[rgb]{0.25,0.63,0.44}{#1}}
\newcommand{\DocumentationTok}[1]{\textcolor[rgb]{0.73,0.13,0.13}{\textit{#1}}}
\newcommand{\ErrorTok}[1]{\textcolor[rgb]{1.00,0.00,0.00}{\textbf{#1}}}
\newcommand{\ExtensionTok}[1]{#1}
\newcommand{\FloatTok}[1]{\textcolor[rgb]{0.25,0.63,0.44}{#1}}
\newcommand{\FunctionTok}[1]{\textcolor[rgb]{0.02,0.16,0.49}{#1}}
\newcommand{\ImportTok}[1]{\textcolor[rgb]{0.00,0.50,0.00}{\textbf{#1}}}
\newcommand{\InformationTok}[1]{\textcolor[rgb]{0.38,0.63,0.69}{\textbf{\textit{#1}}}}
\newcommand{\KeywordTok}[1]{\textcolor[rgb]{0.00,0.44,0.13}{\textbf{#1}}}
\newcommand{\NormalTok}[1]{#1}
\newcommand{\OperatorTok}[1]{\textcolor[rgb]{0.40,0.40,0.40}{#1}}
\newcommand{\OtherTok}[1]{\textcolor[rgb]{0.00,0.44,0.13}{#1}}
\newcommand{\PreprocessorTok}[1]{\textcolor[rgb]{0.74,0.48,0.00}{#1}}
\newcommand{\RegionMarkerTok}[1]{#1}
\newcommand{\SpecialCharTok}[1]{\textcolor[rgb]{0.25,0.44,0.63}{#1}}
\newcommand{\SpecialStringTok}[1]{\textcolor[rgb]{0.73,0.40,0.53}{#1}}
\newcommand{\StringTok}[1]{\textcolor[rgb]{0.25,0.44,0.63}{#1}}
\newcommand{\VariableTok}[1]{\textcolor[rgb]{0.10,0.09,0.49}{#1}}
\newcommand{\VerbatimStringTok}[1]{\textcolor[rgb]{0.25,0.44,0.63}{#1}}
\newcommand{\WarningTok}[1]{\textcolor[rgb]{0.38,0.63,0.69}{\textbf{\textit{#1}}}}
\setlength{\emergencystretch}{3em} % prevent overfull lines
\providecommand{\tightlist}{%
  \setlength{\itemsep}{0pt}\setlength{\parskip}{0pt}}
\setcounter{secnumdepth}{-\maxdimen} % remove section numbering
\ifLuaTeX
  \usepackage{selnolig}  % disable illegal ligatures
\fi
\IfFileExists{bookmark.sty}{\usepackage{bookmark}}{\usepackage{hyperref}}
\IfFileExists{xurl.sty}{\usepackage{xurl}}{} % add URL line breaks if available
\urlstyle{same}
\hypersetup{
  pdftitle={P1 - Serie 01},
  hidelinks,
  pdfcreator={LaTeX via pandoc}}

\title{P1 - Serie 01}
\author{}
\date{}

\begin{document}
\maketitle

Lukas Batschelet (16-499-733)

\subsection{Theorieaufgaben}\label{theorieaufgaben}

\subsubsection{1. Bezeichner}\label{bezeichner}

Aufgabe

Finden Sie passende Bezeichner für...

\begin{itemize}
\tightlist
\item
  eine Java Klasse, die eine Prüfung repräsentieren soll.
\item
  die erreichten Punkte in einer Prüfung.
\item
  eine Methode, welche den Durchschnittswert aller Prüfungen berechnet.
\item
  die maximale Punktzahl, die in einer Prüfung erreicht werden kann.
\end{itemize}

\paragraph{Lösung}\label{luxf6sung}

\begin{Shaded}
\begin{Highlighting}[]
\NormalTok{1. public class Exam \{\}}
\NormalTok{2. int points}
\NormalTok{3. meanPoints()}
\NormalTok{4. final int MAX\_POINTS}
\NormalTok{Copy}
\end{Highlighting}
\end{Shaded}

\subsubsection{2. Variablen und
Eigenschaften}\label{variablen-und-eigenschaften}

Aufgabe

Finden Sie für die Klasse \texttt{Flight} mindestens drei Variablen
(Eigenschaften) und drei Methoden (Verhalten/Funktionen), die in den
Klassen modelliert werden könnten.

\paragraph{Mögliche Lösung}\label{muxf6gliche-luxf6sung}

\subparagraph{2.1 Variablen}\label{variablen}

\begin{enumerate}
\tightlist
\item
  \texttt{altitude}
\item
  \texttt{cargoWeight}
\item
  \texttt{fuelRemaining}
\end{enumerate}

\subparagraph{2.2 Methoden}\label{methoden}

\begin{enumerate}
\tightlist
\item
  \texttt{bookSeat()}
\item
  \texttt{cancelFlight()}
\item
  \texttt{readFuelLevel()}
\end{enumerate}

\subsubsection{3. Zitat}\label{zitat}

Aufgabe

Schreiben Sie eine einzige \texttt{println} Anweisung, die die folgende
Zeichenkette als Ausgabe generiert:

\begin{verbatim}
"Mein Name ist Winston Wolfe.  
Ich löse Probleme!", stellte er sich vor.
Copy
\end{verbatim}

\paragraph{Lösung}\label{luxf6sung-1}

\begin{Shaded}
\begin{Highlighting}[]
\NormalTok{System.out.println("\textbackslash{}"Mein Name ist Winston Wolfe. \textbackslash{}n" +}
\NormalTok{    "Ich löse Probleme!\textbackslash{}", stellte er sich vor.");}
\NormalTok{Copy}
\end{Highlighting}
\end{Shaded}

\subsubsection{4. Rechnung}\label{rechnung}

Aufgabe

Welchen Wert enthält die Variable \texttt{result}, nachdem folgende
Anweisungen durchgeführt worden sind?

\begin{Shaded}
\begin{Highlighting}[]
\NormalTok{int result = 25;  }
\NormalTok{result = result + 5;  }
\NormalTok{result = result / 7;  }
\NormalTok{result = result * 3;  }
\NormalTok{Copy}
\end{Highlighting}
\end{Shaded}

\paragraph{Lösung}\label{luxf6sung-2}

\begin{Shaded}
\begin{Highlighting}[]
\NormalTok{int result = 25;}
\NormalTok{result = result + 5; //30}
\NormalTok{result = result / 7; //4}
\NormalTok{result = result * 3; //12}
\NormalTok{Copy}
\end{Highlighting}
\end{Shaded}

\subsubsection{5. Rechnung}\label{rechnung-1}

Aufgabe

Welchen Wert enthält die Variable \texttt{result}, nachdem folgende
Anweisungen durchgeführt worden sind?

\begin{Shaded}
\begin{Highlighting}[]
\NormalTok{int result = 15, total = 100, min = 15, num = 10;  }
\NormalTok{result /= (total {-} min) \% num;  }
\NormalTok{Copy}
\end{Highlighting}
\end{Shaded}

\paragraph{Lösung}\label{luxf6sung-3}

\begin{Shaded}
\begin{Highlighting}[]
\NormalTok{int result = 15, total = 100, min = 15, num = 10;}
\NormalTok{result /= (total – min) \% num;}
\NormalTok{result = result / ((total – min) \% num;}
\NormalTok{3 = 15 / (( 100 {-} 15) \% 10;}
\NormalTok{Copy}
\end{Highlighting}
\end{Shaded}

\subsubsection{6. Operationen}\label{operationen}

Aufgabe

Gegeben seien folgende Deklarationen:

\begin{Shaded}
\begin{Highlighting}[]
\NormalTok{int result1, num1 = 27, num2 = 5;  }
\NormalTok{double result2, num3 = 12.0;  }
\NormalTok{Copy}
\end{Highlighting}
\end{Shaded}

Welches Resultat wird jeweils durch folgende Anweisungen gespeichert?

\begin{enumerate}
\tightlist
\item
  \texttt{result1\ =\ num1\ /\ num2;}
\item
  \texttt{result2\ =\ num1\ /\ num2;}
\item
  \texttt{result2\ =\ num3\ /\ num2;}
\item
  \texttt{result1\ =\ (int)\ num3\ /\ num2;}
\item
  \texttt{result2\ =\ (double)\ num1\ /\ num2;}
\end{enumerate}

\paragraph{Lösung}\label{luxf6sung-4}

\begin{Shaded}
\begin{Highlighting}[]
\NormalTok{int result1, num1 = 27, num2 =5;}
\NormalTok{double result2, num3 = 12.0;}
\NormalTok{a. result1 = num1 / num2; //5.0}
\NormalTok{b. result2 = num1 / num2; //5.4}
\NormalTok{c. result2 = num3 / num2; //2.4}
\NormalTok{d. result1 = (int) num3 / num2; //2}
\NormalTok{e. result2 = (double) num1 / num2; //5.4}
\NormalTok{Copy}
\end{Highlighting}
\end{Shaded}

\subsubsection{7. Boolsche Operationen}\label{boolsche-operationen}

Aufgabe

Gegeben seien folgende Deklarationen:

\begin{Shaded}
\begin{Highlighting}[]
\NormalTok{int val1 = 15, val2 = 20;  }
\NormalTok{boolean ok = false;  }
\NormalTok{Copy}
\end{Highlighting}
\end{Shaded}

Was ist der Wert der folgenden Booleschen Ausdrücke?

\begin{enumerate}
\tightlist
\item
  \texttt{val1\ \textless{}=\ val2}
\item
  \texttt{(val1\ +\ 5)\ \textgreater{}=\ val2}
\item
  \texttt{val1\ \textless{}\ val2\ /\ 2}
\item
  \texttt{val1\ !=\ val2}
\item
  \texttt{!(val1\ ==\ val2)}
\item
  \texttt{(val1\ \textless{}\ val2)\ \textbar{}\textbar{}\ ok}
\item
  \texttt{(val1\ \textgreater{}\ val2)\ \textbar{}\textbar{}\ ok}
\item
  \texttt{(val1\ \textless{}\ val2)\ \&\&\ !ok}
\item
  \texttt{ok\ \textbar{}\textbar{}\ !ok}
\end{enumerate}

\paragraph{Lösung}\label{luxf6sung-5}

\begin{Shaded}
\begin{Highlighting}[]
\NormalTok{int val1 = 15, val2 = 20;}
\NormalTok{Boolean ok = false;}
\NormalTok{a. true}
\NormalTok{b. true}
\NormalTok{c. false}
\NormalTok{d. true}
\NormalTok{e. true}
\NormalTok{f. true}
\NormalTok{g. false}
\NormalTok{h. true}
\NormalTok{i. true}
\NormalTok{Copy}
\end{Highlighting}
\end{Shaded}

\subsection{Implementationsaufgaben}\label{implementationsaufgaben}

\subsubsection{\texorpdfstring{1. Einfache Ausgabe -
\texttt{WinterIsComing.java}}{1. Einfache Ausgabe - WinterIsComing.java}}\label{einfache-ausgabe---winteriscoming.java}

Aufgabe

Schreiben Sie ein Programm, welches den Satz "Winter is coming" ausgibt
(erste Version: auf einer Zeile; zweite Version: jedes Wort auf einer
separaten Zeile).

\paragraph{Lösung}\label{luxf6sung-6}

\begin{Shaded}
\begin{Highlighting}[]
\NormalTok{public class WinterIsComing \{}
\NormalTok{    public static void main(String[] args) \{}
\NormalTok{        System.out.println("\textbackslash{}"Winter is coming\textbackslash{}"");}
\NormalTok{        System.out.println("\textbackslash{}"Winter \textbackslash{}n" +}
\NormalTok{                "is \textbackslash{}n" +}
\NormalTok{                "coming\textbackslash{}"");}
\NormalTok{    \}}
\NormalTok{\}}
\NormalTok{Copy}
\end{Highlighting}
\end{Shaded}

\subsubsection{\texorpdfstring{2. Einfache Berechnungen -
\texttt{Quotient.java}}{2. Einfache Berechnungen - Quotient.java}}\label{einfache-berechnungen---quotient.java}

Aufgabe

Schreiben Sie ein Programm, das vom Benutzer die Eingabe von zwei
ganzzahligen Werten \texttt{a} und \texttt{b} fordert. Ihr Programm soll
den Quotienten {} sowohl als Gleitkommazahl (d.h. ungerundet) als auch
als ganze Zahl mit Rest berechnen und beide Ergebnisse am Bildschirm
ausgeben. Testen Sie Ihr Programm mit beliebigen Zahlen.

Beobachten Sie insbesondere das Programmverhalten bei Eingabe der Zahl
\texttt{0} als Divisor und versuchen Sie diesen Laufzeitfehler
abzufangen.

\paragraph{Mögliche Lösung}\label{muxf6gliche-luxf6sung-1}

\begin{Shaded}
\begin{Highlighting}[]
\NormalTok{import java.util.Scanner;}

\NormalTok{public class Quotient \{}
\NormalTok{    public static void main(String[] args) \{}

\NormalTok{        System.out.println("Dieses Programm berechnet den Quotienten zweier Zahlen \textbackslash{}"a\textbackslash{}" und \textbackslash{}"b\textbackslash{}".");}
        
\NormalTok{        Scanner scan = new Scanner(System.in);}
\NormalTok{        System.out.println("Geben Sie den Teil \textbackslash{}"a\textbackslash{}" ein:");}
\NormalTok{        double var1 = scan.nextDouble();}
        
\NormalTok{        System.out.println("Geben Sie den Teil \textbackslash{}"b\textbackslash{}" ein:");}
\NormalTok{        double var2 = scan.nextDouble();}
        
\NormalTok{        if (var2 != 0) \{ //Wert 0 führt zu divide by zero}
\NormalTok{            double quotientDouble = (var1 * var1) / var2;}
\NormalTok{            // int quotientInt = (int) var1 * (int) var1) / (int) var2;}
\NormalTok{            int quotientInt = (int) quotientDouble;}
\NormalTok{            System.out.println("\_\_\_\_\_\_\_\_\_\_\_\_\_\_\_\_\_\_\_\_\_\_\_\_\_\_\_\_\_\_\_\_\_\_\_\_\_\_\_\_\_\textbackslash{}n" + }
\NormalTok{                "Der Quotient Ihrer Zahlen: \textbackslash{}t" + quotientDouble + }
\NormalTok{                "\textbackslash{}nUnd als \textbackslash{}"int\textbackslash{}":\textbackslash{}t \textbackslash{}t \textbackslash{}t" + quotientInt);}
\NormalTok{        \} else \{ }
\NormalTok{            System.out.println("Geben Sie nicht 0 ein!");}
\NormalTok{        \}}
\NormalTok{        scan.close();}
\NormalTok{    \}}
\NormalTok{\}}
\NormalTok{Copy}
\end{Highlighting}
\end{Shaded}

\subsubsection{\texorpdfstring{3. Benutzerinteraktion -
\texttt{HumanThermometer.java}}{3. Benutzerinteraktion - HumanThermometer.java}}\label{benutzerinteraktion---humanthermometer.java}

Aufgabe

Schreiben Sie ein Programm, das vom Benutzer die Eingabe einer
Temperatur \texttt{t} fordert. Die Ausgabe Ihres Programmes definiert
sich danach folgendermassen:

Hinweis: Verwenden Sie Konstanten für beide Temperaturgrenzen.

\paragraph{Mögliche Lösung}\label{muxf6gliche-luxf6sung-2}

\begin{Shaded}
\begin{Highlighting}[]
\NormalTok{import java.util.Scanner;}

\NormalTok{public class HumanThermometer \{}
\NormalTok{    public static void main(String[] args) \{}
\NormalTok{        final int LOWER\_BOUND = 15;}
\NormalTok{        final int UPPER\_BOUND = 24;}
        
\NormalTok{        Scanner scan = new Scanner(System.in);}
\NormalTok{        System.out.println("Die aktuelle Temperatur: ");}
\NormalTok{        int temperature = scan.nextInt();}
        
\NormalTok{        if (temperature \textless{} LOWER\_BOUND) \{}
\NormalTok{            System.out.println("Es ist kalt");}
\NormalTok{        \} }
\NormalTok{        else if ((LOWER\_BOUND \textless{}= temperature) \&\& (temperature \textless{}= UPPER\_BOUND)) \{}
\NormalTok{            System.out.println("Es ist angenehm");}
\NormalTok{        \} }
\NormalTok{        else}
\NormalTok{            System.out.println("Es ist warm");}
        
\NormalTok{        scan.close();}
\NormalTok{    \}}
\NormalTok{\}}
\NormalTok{Copy}
\end{Highlighting}
\end{Shaded}


\end{document}
